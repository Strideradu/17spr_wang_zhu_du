% !TEX root = /Users/zhuzhuangdi/Desktop/MSUCourses/MachineLearning847/17Project/17spr_wang_zhu_du/Middle/middle_report.tex
\begin{abstract}  
Song Ci is a precious cultural heritage of China, which is various in styles and sophisticated in syntactic rules.
%
In this project, we develop a system to automatically generate Chinese Song Ci using Recurrent Neural Network (RNN).
%
We use a vector space model to convert each Chinese character as a vector which still reserves the semantic relevance among different characters.
%
Then we use the vector presentation as input to train the RNN model.
%
We also implement Long Short Term Memory units (LSTMs) into the RNN model to learn the semantic meaning in long-distance.
  %
We will  compare the performance of this model with the traditional genetic approaches used to generate poems. 
%
We hope that our system can learn the complete rule from training corpus without any given constraints, and can generate elegant Song Ci poems that follow syntax rules.
% 

\end{abstract}
\keywords{Song Ci (poetry); Recurrent Neural Network; Long Short Term Memory}