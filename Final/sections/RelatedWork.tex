% !TEX root = /Users/zhuzhuangdi/Desktop/MSUCourses/MachineLearning847/17spr_wang_zhu_du/Final/final_report.tex
\section{Related Work}   
Approaches to poetry automatic generation can be divided into the following categories.

\textbf{Using rules and templates.}
%
This approach adopts templates to generate poems that comply with a set of rules and constraints.
%
These rules may be derived from user queries and additional lexical resources\cite{wu2009new,tosa2008hitch},  parts of speech and WordNet patterns \cite{netzer2009gaiku},  or from semantic and grammar templates \cite{oliveira2012poetryme}.
% 
Some other approaches consider the poetry generation as an optimization problem based on a summarization framework with several constraints \cite{yan2013poet}.

\textbf {Using genetic algorithms.} This approach is mainly based on natural selection. It generates all possible candidates, and use search and evaluation algorithms to select the optimal one\cite{manurung2004evolutionary,manurung2012using}.
%
For example, Zhou et al.\cite{zhou2010genetic} proposed an approach in 2016 to automatically generate Song Ci by genetic algorithm.
%
They implemented a fitness function that is designed for evaluate the performance of Song Ci by considering the level and oblique tones-based coding method and the syntactic and semantic correctness.


\textbf{Using Statistical Machine Translation (SMT) methods.} 
 This approach first receives keywords and extract most relevant constituents to theses keywords.  Next, it generates poems by iteratively selecting among these constituents based on phonological, structural, and poetic requirements \cite{jiang2008generating}.  

\textbf{Using neural network.}  Recently, approach using neural network have achieved great success in poem generation, which considers the poetry generation as a sequence-to-sequence generation problem.
% 
In general, this kind of approach will generate new sentence based on previously generated content, so that the generated sentence can capture the semantic consistency automatically \cite{wang2016chinese,bahdanau2014neural}.
%
For example, Zhang \etal built a quatrain generation model using a Recurrent Neural Network model.  Their model generates the first line of the poetry from the given keywords, and then generate subsequent lines by backtracking the status of the lines previously generated.
% 

\textbf{Using Generative Adversarial Networks(GANs) } 

GANs is a popular deep generative model that paring a generator networks and a discriminator networks\cite{goodfellow2014gan}. Most GANs works focus on the computer vision and graphics area, like colorization or image construction. GANs Model based on RNNs can be used to generate musics\cite{mogren2016crnngan} and texts\cite{yu2016seqgan}. In Yu \etal paper, they also generate poem to serve as one example to evaluate the performance of their model. We want to extend this work by train with dataset of Song Ci to further evaluate its performance.