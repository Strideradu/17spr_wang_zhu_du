% !TEX root = /Users/zhuzhuangdi/Desktop/MSUCourses/MachineLearning847/17Project/17spr_wang_zhu_du/Proposal/main.tex
\section{Problem Description}
\subsection{Motivation}
In this project, we propose and evaluate different approaches to automatically generate Chinese poems. 
%
Especially, we study how to automatically generate Chinese Ci using machine learning skills.
%
Ci are one of the most important genres of Chinese classical poetry. 
%
As a precious cultural heritage, not many of them have been passed down onto the current generation.
%
Therefore, the study of automatic generation of Ci is meaningful, not only because it supplements entertainment and education resources to modern society, but also because it demonstrates the feasibility of applying artificial intelligence in Art generation. 
%

\subsection{Background}
Ci is a form of Chinese classical poetry. Arisen with the so-called banquet music in Tang dynasty. It reached its peak about hundred years later, and became a major alternative to Shi poetry\cite{cai2008chinesepoetry} in the Song dynasty.

Although different from even line structure used in Shi, Ci still follows strict rule determining the number of characters for different lines, the arrangement of rhyme, and the location of tones. There are more than 800 rule sets for Ci, which is called Cipai\cite{wikici}. The author of Ci needs to filling in the words according to the matrix associated to the Cipai. The uneven line used empower Ci to use more continuous syntax that traditional shi\cite{cai2008chinesepoetry}.

\subsection{Proposed Approach}
We propose an AI system which generate Ci in an interactive approach.
%
First, our system will prompt the user to provide a Cipai name.
%
Because Ci belonging to different Cipai may contain different emotions or grammatical rules.
%
Next, the system will receive few of keyword inputs that convey the detailed sentiments of the Song Ci.
%
the first sentence of the iambic will be generated based on the keyword inputs.
%
Further, the system generate following sentences based on previously-generated contexts using both RNN and SMT technique.
%
Finally, we evaluate the quality of the generated Ci using an evaluation tool named BLEU.

\subsection{Technical Challenges and Proposed Solutions}
The first challenge to build a general model for all types of Song Ci.
%
Different from Shi poetry whose structure is strict,  Song Ci has more than 800 set of Cipai, and different Cipai follows different structural or rhythmic patterns.
%
Therefore, it is difficult to generalize a model for all the Song Ci from limited training dataset.
% 
Our solution is to create a model based on Recurrent Neural Network. For every line generated in the SongCi, its probability is based on the probability of all previously lines.

Another challenge is to maintain consistent and poetic meanings throughout the generated SongCi.
%
Compared with Shi poetry, Song Ci are much longer in length and therefore more complicated in context.
%
It is difficult to keep long-distance memory using conventional RNN.
% 
Our solution is to use a Long Short Term Memory (LSTM) model that can track the long-distance information. 