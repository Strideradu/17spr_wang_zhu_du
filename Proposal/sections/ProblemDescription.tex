\section{Problem Description}
\subsubsection{Background}
Ci is a form of poetry, arisen with the so called banquet music in Tang dynasty. After about hundred years, it reached its peak in Song dynasty and became a major alternative to shi poetry\cite{cai2008chinesepoetry}.\\

Although unlike the even line structure used in shi, ci is still has strict rule determining number of characters for different lines, the arrangement of rhyme, and the location of tones. There are more than 800 rule sets for ci, which is called cipai\cite{wikici}. The author of ci need to filling in the words according to the matrix associated to the cipai. The uneven line used empower ci to use more continuous syntax that traditional shi\cite{cai2008chinesepoetry}.
\subsubsection{Motivation}
In this project, we propose and evaluate different approaches to automatically generate Chinese poems. 
%
Especially, we study how to automatically generate Chinese Song Iambics using machine learning skills.
%
Song iambics are one of the most important genres of Chinese classical poetry. 
%
As a precious cultural heritage, not many of them have been passed down onto the current generation.
%
Therefore, the study of automatic generation of Song iambics is meaningful, not only because it supplements entertainment and education resources to modern society, but also because it demonstrates the feasibility of applying artificial intelligence in Art generation. 
%
\subsubsection{Proposed Approach}
We propose an AI system which generate Song iambics in an interactive approach.
%
First, our system will prompt the user to provide a tune name.
%
Because Song Iambics have various tune names, and iambics belong to different tune names may contain different emotions or grammatical rules.
%
Next, the system will receive few of keyword inputs that convey the detailed sentiments of the iambic.
%
the first sentence of the iambic will be generated based on the keyword inputs.
%
Further, the system generate following sentences based on previously-generated contexts using both RNN and SMT technique.
%
Finally, we evaluate the quality of the generated Song iambics using an evaluation tool named BLEU.

\subsubsection{Technical Challenges}
The first challenge is that Song iambics follow structure,
%Compared with Chinese couplets and quatrains, Song iambics are longer in structure, more flexible in sentence length, and have more rhythmic and tonal patterns in one iambic.
The second challenge is that .

 %rhythm rules, 
%The second challenge is that and poetic meanings.

%The third challenge is that.
 