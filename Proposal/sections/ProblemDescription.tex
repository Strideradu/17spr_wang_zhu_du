\section{Problem Description}
\subsection{Background}
Ci is a form of poetry, arisen with the so-called banquet music in Tang dynasty. After about hundred years, it reached its peak in Song dynasty and became a major alternative to Shi poetry\cite{cai2008chinesepoetry}.\\

Although unlike the even line structure used in Shi, Ci is still had strict rule determining the number of characters for different lines, the arrangement of rhyme, and the location of tones. There are more than 800 rule sets for Ci, which is called Cipai\cite{wikici}. The author of Ci needs to filling in the words according to the matrix associated to the Cipai. The uneven line used empower Ci to use more continuous syntax that traditional shi\cite{cai2008chinesepoetry}.
\subsection{Motivation}
In this project, we propose and evaluate different approaches to automatically generate Chinese poems. 
%
Especially, we study how to automatically generate Chinese Ci using machine learning skills.
%
Ci are one of the most important genres of Chinese classical poetry. 
%
As a precious cultural heritage, not many of them have been passed down onto the current generation.
%
Therefore, the study of automatic generation of Ci is meaningful, not only because it supplements entertainment and education resources to modern society, but also because it demonstrates the feasibility of applying artificial intelligence in Art generation. 
%
\subsection{Proposed Approach}
We propose an AI system which generate Ci in an interactive approach.
%
First, our system will prompt the user to provide a tune name.
%
Because Ci have various tune names, and Ci belong to different tune names may contain different emotions or grammatical rules.
%
Next, the system will receive few of keyword inputs that convey the detailed sentiments of the iambic.
%
the first sentence of the iambic will be generated based on the keyword inputs.
%
Further, the system generate following sentences based on previously-generated contexts using both RNN and SMT technique.
%
Finally, we evaluate the quality of the generated Ci using an evaluation tool named BLEU.

\subsection{Technical Challenges}
The first challenge is that Ci structure is not that strict compared to Shi. And given that there are more 800 set of Cipai, which means it may not easy to generate a general rule for all the Ci from limited training dataset.\\
%Compared with Chinese couplets and quatrains, Song iambics are longer in structure, more flexible in sentence length, and have more rhythmic and tonal patterns in one iambic.

The second challenge is that we need to extract poetic meaning from training sample. And after that, we also need to generate meaningful continuous syntax.

 %rhythm rules, 
%The second challenge is that and poetic meanings.

%The third challenge is that.