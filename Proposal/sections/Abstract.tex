% !TEX root = /Users/zhuzhuangdi/Desktop/mobisys2017/main.tex
\begin{abstract}

In this paper, we study the side-channel information leakage from a laptop to a nearby mobile device.
%
Different from previous work that uses customized hardware to sense electromagnetic (EM) emissions,
%
we present  \emph{MagDetector}, which uses a mobile device to detect and recognize applications running on a laptop by exploiting EM side-channel information leakage from the laptop's CPU.
% 
To detect the launching of an application, we use two detection models based on machine learning techniques.
%
To recognize an launched application, we extract a time-variant feature vector for each application's EM signal based on short time fourier transform and principal component analysis.
%
After we recognize the launched application, we can even recognize different user operations inside the application based on wavelet multi-resolution analysis.  
%
 We implemented \emph{MagDetector} using a smartphone and evaluated it against 10 applications running on a MacBook laptop.
 %
\emph{MagDetector} detected the 10 applications with precision greater than 96\%, and recall greater than 93\%, respectively.
%
It classified the 10 applications with an average accuracy greater than 98\%.
 %
For user operation recognition, it classified 50 webpages opened in a web-browser with an average accuracy greater than 84\%.
 
 
\end{abstract}
\keywords{Side Channel Attack; Magnetometer; Commodity Mobile Device.}