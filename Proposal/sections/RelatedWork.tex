 % !TEX root = /Users/zhuzhuangdi/Desktop/MSUCourses/MachineLearning847/17Project/17spr_wang_zhu_du/Proposal/main.tex
\section{Related Work}  
Approaches to poetry automatic generation can be divided into the following categories
\begin{itemize}
\item Approaches using rules and templates. This method meets requirements grammaticality, but perform poorly on meaningfulness and poeticness.
\item Approaches using evolutionary algorithms. Natural selection.
\item Another approach is based on the methods for the generation of other kinds of texts.
statistical machine translations (SMT).
\item Approaches using neural network.  Wang \etal adopted an RNN Encoder-Decoder structure\cite{wang2016chinese}. They generate new iambics context using previously-generated contexts. The rationale of their approach is that, in Chinese poems, two consecutive liens have high semantical relevance. They use attention mechanism \cite{bahdanau2014neural}.

We train the model to generate Song Iambics based on Chinese characters.
\end{itemize}
