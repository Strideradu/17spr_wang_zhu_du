% !TEX root = /Users/zhuzhuangdi/Desktop/MSUCourses/MachineLearning847/17Project/17spr_wang_zhu_du/Proposal/main.tex
\section{Proposal Summary and Project Milestones}
In this project, we aim at implementing an automatic Song Ci generator to generate Song Ci poems that can satisfy grammar, rhythmic and poetic requirements.
%
We implement this generator using two approaches. First, we will implement it using recurrent neural networks.
%
Next, we would like to implement it with using genetic algorithms.
%
We will compare the performance of different generating algorithms.
%
We plan to complete our project in the following four steps:
%and detailed project timeline is showed in Figure \ref{fig:projecttimeline}:

\begin{itemize}
\item Background survey
\item Corpus search and selection
\item Algorithm implements
\item Algorithm testing and comparison
\end{itemize}



For this initial step, we plan to search for some related works of computational literary creation and gain insight into the basic knowledge of Song Ci, for example, what is thse criterion of a good Song Ci, how to evaluate the correctness, fluency and style of poems generated. Better understanding of related work and Song Ci composition rules will provide us great help for the following work, especially algorithm testing and comparison. Then, we will search and select a proper Song poem corpus for our project. The ideal corpus should be comprehensive on poem styles, and are precisely analyzed for content. Implementation poem generator based on both algorithm of RNN and genetic algorithm would be the most important work in our project. So we allocate more time on this step. Then test cases will be generated with both poem generator under the same keywords and topics. Poems will be test on aspects of grammar, semantic correctness, style and content. Both computational evaluation and human evaluation are expected to be used in last part of our project.
%\begin{figure*}[t]
%	\centering
%	\includegraphics[width=0.8\textwidth]
%	{ProposedTimeline.png}
%	\caption{Time line of project}
%	\label{fig:projecttimeline}	
%\end{figure*}

