% !TEX root = /Users/zhuzhuangdi/Desktop/MSUCourses/MachineLearning847/17Project/17spr_wang_zhu_du/Middle/middle_report.tex
\section{Related Work}  
\judy{To be extened} 
%
Approaches to poetry automatic generation can be divided into the following categories.

\textbf{Using rules and templates.}
%
This approach adopts templates to generate poems that comply with grammatical rules, such as the rhythms, lines, and word frequencies\cite{wu2009new,tosa2008hitch}. 
%However, this approach performs poorly on semantical and poetic requirements.

\textbf {Using evolutionary algorithms.} This approach is mainly based on natural selection. It generates all possible candidates, and use search and evaluation algorithms to select the optimal one\cite{manurung2004evolutionary,manurung2012using}.
%

\textbf{Using Statistical Machine Translation (SMT) methods.}
%A similar study was conducted on the rap lyrics\cite{malmi2015dopelearning}.Feature related to rhyming, structure similarity, and semantic similarity were used fro rankSVM algorithm. %The algorithm will try to generate a linear model that can give two lines from a lyrics a relevance score. The score is used to generate the prediction of next line lyrics from a given line of lyrics\cite{manurung2004evolutionary}.
\cite{jiang2008generating}. This approach first receives keywords and extract most relevant constituents to theses keywords.  Next, it generates poems by iteratively selecting among these constituents based on phonological,
structural, and poetic requiremetns.

\textbf{Using neural network.}  This approach adopts an RNN Encoder-Decoder structure \cite{wang2016chinese,bahdanau2014neural}. 
%
It generates new iambics context using previously-generated contexts. based on the rationale  that, in Chinese poems, two consecutive lines have high semantical relevance.
  