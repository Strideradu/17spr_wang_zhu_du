% !TEX root = /Users/zhuzhuangdi/Desktop/MSUCourses/MachineLearning847/17Project/17spr_wang_zhu_du/Middle/middle_report.tex
\section{Data Description}   
For our experiment, we obtained dataset for both Tang Shi and Song Ci. Many research were conducted for automatically generating Tang Shi. So we can evaluate our experiment result by comparing with these machine-created Tang Shi. And then we can move forward to Song Ci.
\subsubsection{Tang Poetry Corpus}
We use Quan Tangshi as our Tang Poetry corpus.\cite{1960quantangshi}. It was commissioned by Yin Cao in 1705 and published under the name of Kangxi Emperor. It contains 49,000 lyric poems (in the dataset we used it has 49,274 poems) and is believed the largest collection of Tang poetry. We obtained the dataset from the server of \cite{zhang2014chinese}.
\subsubsection{Song Ci Corpus}